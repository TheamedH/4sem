
\documentclass[danish,12pt,a4paper]{article}

\usepackage[utf8]{inputenc}
\usepackage[danish]{babel}					% Dokumentets sprog
\usepackage[T1]{fontenc}
\usepackage[dvipsnames]{xcolor}
\let\proof\relax
\let\endproof\relax
\usepackage{amsthm}
\usepackage{amsmath,amsthm,amssymb,amsfonts,bm}
\usepackage{ragged2e,anyfontsize}
\usepackage{etex}
\usepackage{nicefrac,xfrac}
\usepackage[titletoc]{appendix}
\usepackage{thmtools}
\usepackage[pagebackref]{hyperref}
\usepackage{MnSymbol,wasysym}
\usepackage[makeroom]{cancel}
\usepackage{placeins}
\usepackage{multirow}
\usepackage{mathtools}
\usepackage[danish]{varioref}
\usepackage{natbib}
\usepackage{dirtytalk}
\usepackage{blindtext}
\RequirePackage{etex}
\usepackage{graphicx}
\usepackage{flafter}
\usepackage{float}
\usepackage{systeme}
\usepackage{lastpage}
\usepackage{fancyhdr}
\usepackage{etoolbox}
\usepackage{mdframed}
\usepackage{enumitem}
\usepackage[a4paper]{geometry}
\usepackage{a4wide}
\usepackage{parskip}
\usepackage{booktabs}
\usepackage{rotating}
\usepackage{colortbl}
\usepackage{textcomp}
\usepackage{tabularx}
\usepackage{url}
\usepackage{hyperref}
\usepackage[noend]{algorithmic}
\usepackage{tasks}
\usepackage{commath}
\usepackage{calc}
\usepackage{calrsfs}
\usepackage{xfrac}
\usepackage{blkarray}
\usepackage[labelfont=bf]{caption}
\usepackage{mathrsfs}
\usepackage[mathscr]{euscript}
\usepackage{listings}
\usepackage{tcolorbox}
\usepackage{mathdots}
\usepackage{listings}
\usepackage{bbm}

\lstdefinestyle{mystyle}{
    backgroundcolor=\color{white},
    commentstyle=\color{purple},
    keywordstyle=\color{blue},
    numberstyle=\tiny\color{black},
    stringstyle=\color{orange},
    basicstyle=\footnotesize,
    breakatwhitespace=false,
    breaklines=true,
    captionpos=b,
    keepspaces=true,
    numbers=left,
    numbersep=5pt,
    showspaces=false,
    showstringspaces=false,
    showtabs=false,
    tabsize=2
}

\lstset{style=mystyle}
\renewcommand{\labelenumi}{\alph*)}
\renewcommand{\implies}{\Rightarrow}
\renewcommand{\vec}{\overrightarrow}
\newcommand{\vn}{\varnothing}
\newcommand{\an}{$\{a_n\}_{n=1}^{\infty}$}
\newcommand{\zn}{$\{z_n\}_{n=1}^{\infty}$}
\newcommand{\darr}{\Leftrightarrow}
\renewcommand{\to}{\rightarrow}
\newcommand{\la}{\langle}
\newcommand{\ra}{\rangle}
\newcommand{\prtx}{\frac{\partial f}{\partial x}}
\newcommand{\prty}{\frac{\partial f}{\partial y}}
\newcommand{\vectwo}[2]{\begin{pmatrix}#1\\#2\end{pmatrix}}

\newcommand{\bigzero}{\mbox{\normalfont\Large\bfseries 0}}
\newcommand{\rvline}{\hspace*{-\arraycolsep}\vline\hspace*{-\arraycolsep}}
% \renewcommand{\Boo}{\phantom{1}}
% \renewcommand{\boo}{\phantom{.}}
% \renewcommand{\Bop}{\phantom{+}}

%Fikser underfull og overfull
\hbadness=10001
\vbadness=10001

%Bibliografi og litteratur
%\bibliographystyle{Formalia/vancouver}
\bibliographystyle{agsm}
\setcitestyle{square}
\usepackage[nottoc,numbib]{tocbibind}


%Gør indholdsfortegnelse og bibliografi dansk
\addto\captionsdanish{
	\renewcommand\contentsname{Indholdsfortegnelse}
	\renewcommand{\bibname}{Bibliografi}
}

% Definer * til at være \cdot %
\mathcode`\*="8000
{\catcode`\*\active\gdef*{\cdot}}

%%%% ORDDELING %%%%
\hyphenation{In-te-res-se e-le-ment}

%Sidehoved
\setlength{\headheight}{15pt}
\pagestyle{fancy}
\fancyhf{}
\newcommand{\chaptermark}[1]{ \markboth{\thechapter.\ #1}{}}
\fancyheadoffset{0pt}
\lhead{\nouppercase \leftmark}
\rhead{Aalborg Universitet}
\renewcommand{\chaptermark}[1]
        {\markboth{#1}{}}
\renewcommand{\sectionmark}[1]
        {\markright{\thesection\ #1}}
\lfoot[\fancyplain{}{\bfseries\thepage}]
    {\fancyplain{}{}}
\rfoot[\fancyplain{}{}]%
    {\fancyplain{}{\bfseries\thepage}}
\patchcmd{\chapter}{plain}{fancy}{}{}

%Kapiteludseende
\usepackage{xcolor}
\usepackage{titlesec, blindtext, color}
\definecolor{gray75}{gray}{0.75}
\newcommand{\hsp}{\hspace{20pt}}
\titleformat{\chapter}[hang]{\huge\bfseries}{\thechapter\hsp\textcolor{gray75}{|}\hsp}{0pt}{\huge\bfseries}
\titlespacing*{\chapter}{0pt}{5pt}{25pt}

% Define a simple command to use at the start of a table row to make it have a shaded background
\newcommand{\gray}{\rowcolor[gray]{.9}}

\usepackage{textcomp}
\usepackage{url}

\usepackage{pst-func}
%TikZ
\usepackage{pgfplots}
\pgfplotsset{compat=1.15}
\usepackage{pgf,tikz}
\usetikzlibrary{arrows,positioning,automata, petri, topaths,graphs,graphs.standard,arrows.meta}
\usepackage{tkz-berge}
%\usepackage[position=top]{subfig}
\usepackage{subcaption}
\usepackage{verbatim}
\usepackage{marvosym}
%---- pseudocode
\usepackage[ruled,linesnumbered]{algorithm2e}
%\usepackage{algorithm}
\floatname{algorithm}{Algoritme}
\renewcommand{\gets}{:=}

% \usepackage{framed}
% \definecolor{myGray}{HTML}{F9F9F9}
% \renewenvironment{leftbar}[4][\hsize]
% {\def\FrameCommand
%     {{\color{#2}\vrule width #4pt}
%         \hspace{-8pt}
%         \fboxsep=\FrameSep\colorbox{#3}}
%     \MakeFramed{\hsize#1\advance\hsize-\width\FrameRestore}}
% {\endMakeFramed}

% %Sætninger, definitioner, mm. general stil
\declaretheoremstyle[
    % spaceabove=18pt,
    % spacebelow=18pt,
    headfont=\normalfont\bfseries,
    bodyfont = \normalfont\itshape,
    postheadspace=2mm,
    headpunct={.}]{mystyle}
\declaretheoremstyle[
    % spaceabove=18pt,
    % spacebelow=18pt,
    headfont=\normalfont\bfseries,
    bodyfont = \normalfont,
    margin = 50,
    postheadspace=2mm,
    headpunct={.}]{mystyle2}


% %Sætning
\declaretheorem[name={Sætning}, style=mystyle,numberwithin=section]{thm}
\newenvironment{thmx}[1]
    {\begin{thm}#1}{\end{thm}}

% %Definition
\declaretheorem[name={Definition}, style=mystyle,sibling=thm]{defni}
\newenvironment{defn}[1]
    {\begin{defni}#1}{\end{defni}}

% %Eksempel
% \declaretheorem[name={Eksempel}, style=mystyle2, sibling=thm]{exmp}
% \newenvironment{eks}[1]
%     {\begin{exmp}#1}{\end{exmp}}
 %Lemma
\declaretheorem[name={Lemma}, style=mystyle,sibling=thm]{lema}
\newenvironment{lem}[1]
    {\begin{lema}#1}{\end{lema}}

%Bevis
\declaretheoremstyle[
    spaceabove=14pt,
    spacebelow=6pt,
    headfont=\normalfont\itshape\bfseries,
    bodyfont = \normalfont,
    postheadspace=1mm,
    qed=$\blacksquare$,
    headpunct={.}]{bevisstyle}
\declaretheorem[name={Bevis}, style=bevisstyle,numbered=no]{bev}

%Inline graphics
\newlength\myheight
\newlength\mydepth
\settototalheight\myheight{Xygp}
\settodepth\mydepth{Xygp}
\setlength\fboxsep{0pt}
\newcommand*\inlinegraphics[1]{%
  \settototalheight\myheight{Xygp}%
  \settodepth\mydepth{Xygp}%
  \raisebox{-\mydepth}{\includegraphics[height=\myheight]{#1}}}


%Kommandoer, som gør jeres liv nemmere, når I skriver. Pas på med at lave for mange kommandoer selv
%da det kan være træls for jer når I skal indsende MWE (minimal working examples) ind i fx stackexchange
\newcommand{\R}{\mathbb{R}}
\newcommand{\Z}{\mathbb{Z}}
\newcommand{\N}{\mathbb{N}}
\renewcommand{\d}{\mathrm{d}}
\newcommand{\eps}{\varepsilon}
\newcommand{\e}{\mathrm{e}}
\newcommand{\E}{\mathcal{E}}
\newcommand{\tr}{\mathrm{tr }}
\newcommand{\F}{\mathbb{F}}
\renewcommand{\L}{\mathcal{L}}
\newcommand{\cross}{\times}
\newcommand{\m}{\cdot}
\makeatletter
\renewcommand*\env@matrix[1][*\c@MaxMatrixCols c]{%
  \hskip -\arraycolsep
  \let\@ifnextchar\new@ifnextchar
  \array{#1}}
\makeatother

% Number of matrix cloumns
\setcounter{MaxMatrixCols}{20}
